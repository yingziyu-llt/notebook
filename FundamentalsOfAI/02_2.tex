% vim:ft=tex:
%
\documentclass[UTF8]{ctexart}
\usepackage{amsmath}
\usepackage{listings}

\lstset{
    basicstyle          =   \sffamily,          % 基本代码风格
    keywordstyle        =   \bfseries,          % 关键字风格
    commentstyle        =   \rmfamily\itshape,  % 注释的风格,斜体
    stringstyle         =   \ttfamily,  % 字符串风格
    flexiblecolumns,                % 别问为什么,加上这个
    numbers             =   left,   % 行号的位置在左边
    showspaces          =   false,  % 是否显示空格,显示了有点乱,所以不现实了
    numberstyle         =   \zihao{-5}\ttfamily,    % 行号的样式,小五号,tt等宽字体
    showstringspaces    =   false,
    captionpos          =   t,      % 这段代码的名字所呈现的位置,t指的是top上面
    frame               =   lrtb,   % 显示边框
}

\lstdefinestyle{Python}{
    language        =   Python, % 语言选Python
    basicstyle      =   \zihao{-5}\ttfamily,
    numberstyle     =   \zihao{-5}\ttfamily,
    keywordstyle    =   \color{blue},
    keywordstyle    =   [2] \color{teal},
    stringstyle     =   \color{magenta},
    commentstyle    =   \color{red}\ttfamily,
    breaklines      =   true,   % 自动换行,建议不要写太长的行
    columns         =   fixed,  % 如果不加这一句,字间距就不固定,很丑,必须加
    basewidth       =   0.5em,
}
\title{
	数学基础
}
\author{
	Letian Lin --- \texttt{yingziyu-Lin@outlook.com}
}

\begin{document}
\maketitle
\section{概率论}
\subsection{概率的定义}
频率学派:一个事情发生的次数占总试验次数的比例

贝叶斯学派:概率是表达个人或主观信念的不确定性

一定是$[0,1]$上的一个数

\subsection{联合概率}
事件$X=x_i$和$Y=y_i$同时发生的概率,记作$P(X=x_i,Y=y_j) = \frac{n_{i,j}}{N}$

加和原则:$P(X=x_i)=\sum^L_{j=1}P(X=x_i,Y=Y_i)$

\subsection{条件概率}
在事件$X=x_i$发生条件下,事件$Y=y_j$发生的概率,记作$P(X=x_i|Y=y_j)=\frac{n_{i,j}}{c_i}$

$p(X,Y)=p(Y|X)*p(X)$

\subsection{贝叶斯定理}
$$P(A|B)=\frac{P(B|A)*P(A)}{P(B)}$$

$P(B|A)$称作似然度(表达了对a的不同设置,观测到数据集有多大的可能性),$P(A|B)$叫做后验概率(观察到B后的概率),P(A)为先验概率,是对a的猜测

b是观察到的数据,a是模型参数。

$$\textbf{posterior=likeihood*prior}$$

\paragraph{推导}
$$P(Y|X)*P(X)=P(Y,X)=P(X,Y)=P(X|Y)*P(Y)$$

\subsection{独立事件}
两个事件互不影响,Y事件不影响X事件$p(X|Y)=p(X),p),p(Y|X)=p(Y),p(X,Y)=p(X)p(Y)$

\subsection{概率密度/累计分布函数}
$p(x)$概率密度函数(PDF)

$P(x)$累计分布函数(CDF)

性质:$p(x)\geq 0$
$$P(z)=\int$$

\subsection{数学期望}
\paragraph{定义}在概率分布$p(x)$下$f(x)$的均值
\paragraph{计算}$E(f)=\sum_xp(x)f(x)$

$$E(f)=\int p(x)f(x)$$

$$E(f)=\frac{1}{N}\sum^N_j=1f(j)$$

\subsection{协方差}
一个变量偏离期望的时候,另一个变量也偏离期望的趋势

\subsection{随机变量的分布}
\paragraph{高斯分布}$$$$(公式后补)

\subsection{学派}
\paragraph{频率学派}用可重复的事件来计量事情发生的可能性
\paragraph
\end{document}
