% vim:ft=tex:
%
\documentclass[UTF8]{ctexart}
\usepackage{amsmath}
\usepackage{listings}

\lstset{
    basicstyle          =   \sffamily,          % 基本代码风格
    keywordstyle        =   \bfseries,          % 关键字风格
    commentstyle        =   \rmfamily\itshape,  % 注释的风格,斜体
    stringstyle         =   \ttfamily,  % 字符串风格
    flexiblecolumns,                % 别问为什么,加上这个
    numbers             =   left,   % 行号的位置在左边
    showspaces          =   false,  % 是否显示空格,显示了有点乱,所以不现实了
    numberstyle         =   \zihao{-5}\ttfamily,    % 行号的样式,小五号,tt等宽字体
    showstringspaces    =   false,
    captionpos          =   t,      % 这段代码的名字所呈现的位置,t指的是top上面
    frame               =   lrtb,   % 显示边框
}

\lstdefinestyle{Python}{
    language        =   Python, % 语言选Python
    basicstyle      =   \zihao{-5}\ttfamily,
    numberstyle     =   \zihao{-5}\ttfamily,
    keywordstyle    =   \color{blue},
    keywordstyle    =   [2] \color{teal},
    stringstyle     =   \color{magenta},
    commentstyle    =   \color{red}\ttfamily,
    breaklines      =   true,   % 自动换行,建议不要写太长的行
    columns         =   fixed,  % 如果不加这一句,字间距就不固定,很丑,必须加
    basewidth       =   0.5em,
}
\title{
	深度神经网络基础
}
\author{
	Letian Lin --- \texttt{yingziyu-Lin@outlook.com}
}

\begin{document}
\maketitle
\section{神经元}
单个神经元就是一个函数:$$\mathbf{z}=\mathbf{w}^T\mathbf{x}+\textbf{b}$$

w是权重,x叫做输入层,z叫做输出层。由于所有输入层都和输出层连接,所以叫做全连接层。b叫做偏置bias

代码里面喜欢写成:$z=xw$,x是行向量,w是列向量

偏置可以让输出上下浮动,更好适应输入值,让决策界面不一定过原点。

多个神经元:可以获得更多维度的输出。

$$z1=x_1w_{11}+x_2w_{21}+x_3w_{31}+b_1;z_2=\dots$$
$$z=xW$$

\section{激活函数-从线性到非线性}
常用激活函数:sigmoid:[0,1],常用于输出概率

tanh:[-1,1],常用于回归任务,梯度比sigmoid大,好收敛

ReLU:将负数输出为0,正数输出原先的值。用于特征提取和简化网络优化。可以预防梯度消失。

Leaky ReLU:在负数区和ReLU有区别-不是0,而是将原值乘以一个很小的数$\alpha$,也可以避免梯度消失的问题。

Softmax:在多个输出的时候,用Softmax来归一化

\section{多层感知器(MLP)-分开分不开的点}

输入层->隐藏层->输出层

提取特征->编码;将特征改为数据->解码

\section{损失函数(Loss Function)-量化预测输出和训练数据输出的误差}

和线性模型类似,常用的也是交叉熵、Softmax、SquireLoss,两个向量之间的某范数\textit{L}1 Loss:MAE平均绝对误差

还有很多损失函数,也是研究中的重要的一个部分。

\section{优化}
\subsection{梯度下降}
给定网络结构$f(x,\theta)$、定义好的损失函数 $L$和训练数据样本,训练网络的过程是
通过更新网络参数$\theta$(比如权重、偏置)以最小化误差的过程。

梯度下降(Gradient descent)是一个更新网络参数的方法,此外依然有很多优化
方法存在,比如BFGS (L-BFGS) 和 conjugate gradient (CG),但这些方法往往需要
很多计算量,因此很少使用。

对于参数$\theta$,总做:$\theta:=\theta - \frac{\partial L}{\partial \theta}$ L : Loss Function

\subsection{反向传播}
实际上就是计算所有参数$\frac{\partial L}{\partial \theta}$

\subsection{SGD}
随机梯度下降:每次不使用所有训练样本,而是随机选出一批(batch)数据更新误差。这批叫做mini-batch,数量叫batch size。通过多次更新,最后可以覆盖整个数据集,叫做一个epoch。

\subsection{自适应学习率}
自动控制学习率,从而快速准确的收敛到鞍点。

\subsection{超参数选择}
必须在验证数据集里面调参,不允许在测试集里面调参。可以调网络结构,loss function,batch size,weight Decay里面的$\lambda$等等。

数据不足的时候,使用cross validation

\section{正则化}
数据增强,增加更多数据

提前停止法

Weight Decay

dropout:在训练的时候随机对隐藏输出置为0.测试和使用的时候不置0,可以认为是把整个网络拆分成好多个小的子网络,测试的时候各个子网络投票决定结果。
\end{document}
