% vim:ft=tex:
%
\documentclass[UTF8]{ctexart}
\usepackage{amsmath}
\usepackage{listings}

\lstset{
    basicstyle          =   \sffamily,          % 基本代码风格
    keywordstyle        =   \bfseries,          % 关键字风格
    commentstyle        =   \rmfamily\itshape,  % 注释的风格,斜体
    stringstyle         =   \ttfamily,  % 字符串风格
    flexiblecolumns,                % 别问为什么,加上这个
    numbers             =   left,   % 行号的位置在左边
    showspaces          =   false,  % 是否显示空格,显示了有点乱,所以不现实了
    numberstyle         =   \zihao{-5}\ttfamily,    % 行号的样式,小五号,tt等宽字体
    showstringspaces    =   false,
    captionpos          =   t,      % 这段代码的名字所呈现的位置,t指的是top上面
    frame               =   lrtb,   % 显示边框
}

\lstdefinestyle{Python}{
    language        =   Python, % 语言选Python
    basicstyle      =   \zihao{-5}\ttfamily,
    numberstyle     =   \zihao{-5}\ttfamily,
    keywordstyle    =   \color{blue},
    keywordstyle    =   [2] \color{teal},
    stringstyle     =   \color{magenta},
    commentstyle    =   \color{red}\ttfamily,
    breaklines      =   true,   % 自动换行,建议不要写太长的行
    columns         =   fixed,  % 如果不加这一句,字间距就不固定,很丑,必须加
    basewidth       =   0.5em,
}
\title{
	编程基础(第二课时)
}
\author{
	Letian Lin --- \texttt{yingziyu-Lin@outlook.com}
}

\begin{document}
\maketitle

\section{Pytorch编程基础}
pytorch是基于python的一个深度学习框架语言。

pyTorch比TensorFlow更加流行,基于动态图而非静态图。支持各种数据集,提供了统一的数据集接口,文档详细。

对比:

numpy数值运算库,提供了矩阵的各种运算功能。PyTorch写法和numpy接近,将Tensor当作矩阵处理。

\subsection{Tensor}
是一个多维矩阵

\begin{lstlisting}[language=Python]
import torch
t = torch.rand(2,2)
\end{lstlisting}

\begin{lstlisting}
torch.from_numpy(array)#从numpy引入
torch.
\end{lstlisting}
\paragraph{Tensor属性}
形状:tensor.shape

数据类型:tensor.dtype

位置:tensor.device

转到gpu上:tensor.gpu()或tensor.cuda()

tensor支持100多操作,在array中的操作基本都可以。

\paragraph{分片和索引}tensor[0,:]即可输出第一行,tensor[0,0:100]第一行前100个(0-99元素)

\paragraph{矩阵拼接}torch.cat在已有维度上拼接,torch.stack是在没有的维度上拼接

\paragraph{矩阵乘法}@是乘法,*是点乘

\paragraph{处理单个值}tensor.item()将单个值的tensor转为一个数字,若处理多个值tensor.item()会报错

\paragraph{原地操作}会修改自身的值,比较鸡肋
\subsection{Module}
后补
\subsection{Autograd}

\subsection{Optimizer}
\subsection{Load/Save}
torch.save(model,"data.pth")

\subsection{DataSet/DataLoader}


\end{document}
