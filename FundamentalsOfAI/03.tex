% vim:ft=tex:
%
\documentclass[UTF8]{ctexart}
\usepackage{amsmath}
\usepackage{listings}

\lstset{
    basicstyle          =   \sffamily,          % 基本代码风格
    keywordstyle        =   \bfseries,          % 关键字风格
    commentstyle        =   \rmfamily\itshape,  % 注释的风格,斜体
    stringstyle         =   \ttfamily,  % 字符串风格
    flexiblecolumns,                % 别问为什么,加上这个
    numbers             =   left,   % 行号的位置在左边
    showspaces          =   false,  % 是否显示空格,显示了有点乱,所以不现实了
    numberstyle         =   \zihao{-5}\ttfamily,    % 行号的样式,小五号,tt等宽字体
    showstringspaces    =   false,
    captionpos          =   t,      % 这段代码的名字所呈现的位置,t指的是top上面
    frame               =   lrtb,   % 显示边框
}

\lstdefinestyle{Python}{
    language        =   Python, % 语言选Python
    basicstyle      =   \zihao{-5}\ttfamily,
    numberstyle     =   \zihao{-5}\ttfamily,
    keywordstyle    =   \color{blue},
    keywordstyle    =   [2] \color{teal},
    stringstyle     =   \color{magenta},
    commentstyle    =   \color{red}\ttfamily,
    breaklines      =   true,   % 自动换行,建议不要写太长的行
    columns         =   fixed,  % 如果不加这一句,字间距就不固定,很丑,必须加
    basewidth       =   0.5em,
}
\title{
	机器学习基础
}
\author{
	Letian Lin --- \texttt{yingziyu-Lin@outlook.com}
}

\begin{document}
\maketitle

\section{定义}
模型在某些任务上基于经验,利用算法的设计和分析,使模型的表现得到提升。

是一个从数据到知识的过程。

几大要素:任务,经验,表现

\section{任务}
有监督学习:分类问题,回归问题... -> 数据带label

无监督学习:聚类,密度估计,降维... -> 数据不带label

半监督学习 -> 数据有带label,也有不带label的

弱监督学习 -> 数据的label级别与任务需要的label不对等,如像素级的任务但数据只有图片级的label

强化学习

\section{经验}

\section{表现}
从一个随机测试数据X,衡量预测值和真是标签之间的远近程度。

loss function

二分类:01loss

回归分析:square loss$loss(Y,f(X))=(f(X) - Y)^2$

\section{线性模型}
可以做离散也可以做连续的,连续值的预测是一个回归问题,离散值的预测是一个分类问题。

线性模型就是学习特征的一种线性组合来预测y,即:$$f(x)=w^Tx+b$$

其中$w$,$b$是要学习的模型参数。$w$反映了各个元素对房价的影响权重。

实际就是对于数据集$D=\{(x_1,y_1),(x_2,y_2)\dots (x_n,y_n)\}$,找到一组$w,b$,使得$f(x)=w^Tx+b$,且$f^*=argmin_f \mathtt{E}[(f(X)-Y)^2]$

用经验均值代替差异,只需要最小化$$\hat{f}=\frac{1}{n}\sum_{i=1}^n(f(x_i)-y_i)^2$$

经过一系列化简,可以得到:

$$J(\beta)=(A\beta-Y)^T(A\beta-Y)$$

其中$A=[x_1,x_2\dots,x_n]^T$,$\beta=(W,b)$,$Y=[y_1,y_2\dots y_n]^T$

对$J(\beta)$对$\beta$求偏导,可以得到$(A^TA)\beta=A^TY$

若$(A^TA)$可逆,则可以直接求解,不可逆的时候需要正则化。

要是$n<p$的话,不可逆,不能用下述方法来做。->样本数量小于特征维度

\subsection{梯度下降法}
可逆的时候:求逆$\rightarrow$算起来太慢。使用梯度下降法。使用条件:$J(\beta)$恰好是个convex的函数。

过程:initial:$$\beta^0$$
update:$$\beta^{k+1}=\beta^t+\frac{\alpha}{2}\frac{\partial J(\beta)}{\partial \beta}|t$$
stop:$$\frac{\partial J(\beta)}{\partial \beta}|t < \epsilon$$

算法参数分析:$\alpha$大:残差大,速度快,可能出现震动。$\alpha$小:速度慢,准确性好

不可逆的时候,没有提供足够的信息来估计每个trait的权重的:1.BoW(bag of words)方法,降维,将大的样本转化为几个特征表示的向量组。

2.正则化
\subsection{正则化}
既然不能提供足够的信息,那就编信息。

假设参数有先验的分布。不妨让参数$\beta$符合高斯分布/拉普拉斯分布......

只需要:$\hat{\beta}_{MAP}=arg\max log(p(X|\beta) + log(p(\beta)))$

经过一些推导:$$\hat\beta = argmin(\sum (Y_i - X_i\beta)+\lambda ||\beta||)$$
$$\hat\beta=(A^TA+\lambda \mathbf{I})^{-1}A^TY$$

容易知道,Ridge Regression得到了一个\textbf{L2}惩罚,画出来一个高维球体;Lasso Regression得到了一个\textbf{L1}惩罚,画出来一个菱形。
从而:Lasso Regression求出的解是一个稀疏的解,从而有利于降低维度。
Ridge Regression求出的$\beta$是较小的值

$$\min(A\beta - Y)^T(A\beta-Y)+\lambda pen(\beta)=\min J(\beta) + \lambda pen(\beta)$$

\end{document}
