% vim:ft=tex:
%
\documentclass[UTF8]{ctexart}
\usepackage{amsmath}

\title{
	07 重积分
}
\author{
	Letian Lin --- \texttt{yingziyu-Lin@outlook.com}
}

\begin{document}
\maketitle
\tableofcontents
\newpage
\section{二重积分的概念和性质}
\subsection{引入}
\paragraph*{一元函数的积分} 一元函数的积分的核心是分割求和,将连续函数分割成大量小的微元,从而便于求和。其公式为$$\int_a^b f(x) = \lim_{n\to\infty}\sum_{i=1}^n f(x_i) \Delta x_i$$
\paragraph*{一元函数积分的性质}

1.线性性质 $\int_a^b f(x)+g(x) = \int_a^b f(x) + \int_a^b g(x)$
          $\int_a^b kf(x)=k\int_a^bf(x)$

2.保号性:若 $\forall x \in D\ f(x)\leq g(x)$ 或仅在有限个点上不满足该条件,都有 $\int_a^b f(x) \leq \int_a^b$

3.可加性:$\int_a^b f(x)=\int_a^c f(x) + \int_c^b f(x)$

4.积分中值定理:若 $f(x)$ 在 $[a,b]$ 上连续 $\exists \xi \in [a,b]$使得 $\int_a^b f(x) = f(\xi)*(b-a)$

\subsection{二重积分}
\paragraph{分割} 我们把一个区域$D$划分为$n$个子区域,分别称作$D_1,D_2\dots D_n$,要求$D_1 \cup D_2 \cup \dots D_n = D$,且都不相交
\paragraph{定义} 设$z=f(x,y)$是定义在有界闭区域$D$上的函数,对其任意分割$\{D_1,D_2,\dots D_n\}$,及任意选择$(x_i,y_i) \in D_i$,称$$\lim_{\lambda\to 0}\sum_{i=1}^{n}f(x_i,y_i)\Delta S_i$$为$f(x,y)$在$D$上的二重积分,记作$\iint_D f(x,y)$dxdy
\paragraph{二重积分的几何意义} 以$z=f(x,y)$为顶面的直柱体的体积

\end{document}
