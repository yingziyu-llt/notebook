% vim:ft=tex:
%
\documentclass[UTF8]{ctexart}
\usepackage{amsmath}
\usepackage{listings}

\lstset{
    basicstyle          =   \sffamily,          % 基本代码风格
    keywordstyle        =   \bfseries,          % 关键字风格
    commentstyle        =   \rmfamily\itshape,  % 注释的风格,斜体
    stringstyle         =   \ttfamily,  % 字符串风格
    flexiblecolumns,                % 别问为什么,加上这个
    numbers             =   left,   % 行号的位置在左边
    showspaces          =   false,  % 是否显示空格,显示了有点乱,所以不现实了
    numberstyle         =   \zihao{-5}\ttfamily,    % 行号的样式,小五号,tt等宽字体
    showstringspaces    =   false,
    captionpos          =   t,      % 这段代码的名字所呈现的位置,t指的是top上面
    frame               =   lrtb,   % 显示边框
}

\lstdefinestyle{Python}{
    language        =   Python, % 语言选Python
    basicstyle      =   \zihao{-5}\ttfamily,
    numberstyle     =   \zihao{-5}\ttfamily,
    keywordstyle    =   \color{blue},
    keywordstyle    =   [2] \color{teal},
    stringstyle     =   \color{magenta},
    commentstyle    =   \color{red}\ttfamily,
    breaklines      =   true,   % 自动换行,建议不要写太长的行
    columns         =   fixed,  % 如果不加这一句,字间距就不固定,很丑,必须加
    basewidth       =   0.5em,
}
\title{
  transformer
}
\author{
	Letian Lin --- \texttt{yingziyu-Lin@outlook.com}
}

\begin{document}
\maketitle

\section{动机}
\subsection{现有模型的缺陷}
\paragraph{CNN}可以用于文本处理,问题:感受有限,不能捕捉全局依赖关系。
\paragraph{RNN}长序列出现梯度消失和梯度爆炸
\paragraph{LSTM和GRU}误差积累问题,效率低下。

\subsection{注意力}
从视觉来看,人的眼睛只有有限的处理能力,而忽略无关的信息。这就是有注意力机制。有空间选择性和特征选择性。是一种很稀缺的资源。

人看东西的时候只关心中央的一小部分,外周的使用高效编码来快速模糊处理。

自注意力和位置编码,利用自注意力机制,可以建立一个长距离的句子内部联系。

\section{自注意力机制}
注意力提示:查询、键和值。

键和值是非自主性提示,查询是自主性提示。

将Key和Query逐个通过注意力汇聚打分,获得两者之间的相似度,给每个值给相关的函数。常见的注意力汇聚:点积和加分。

$$f(x)=\sum^n_{i=1}\alpha(x,x_i)y_i$$

Q,K,V三个是输入的三个线性变换得到的三个空间。$dim( K )= dim( Q )= dim( R ) = D_K$ $Q=Input*W_Q$这三个矩阵将用于计算注意力。

打分函数-缩放点积:用点积计算QK的相似度,得到注意力权重矩阵Z,将它与V相乘就可以得到目标。

缩放的作用:防止某个值过大导致softmax梯度过小。
$$softmax(\frac{Q* K^T}{\sqrt{D_k}})V=Z$$

再将注意力矩阵和V相乘,获得应用注意力汇聚后的V矩阵。

加性注意力:引入更多可学习参数和非线性层,但计算复杂度大。

$$Attention = softmax(tanh(Q*W_Q+K*W_K)*V_a^T)*V$$
$V_a$是可学习的权重向量。

\section{多头记忆力机制}、
就是h个self-attention组合,将原本的内容映射到h各空间,关注不同的特征维度。将这h个注意力汇聚的输出concatinate在一起,然后用一个可学习的线性变化放在一起。

\section{transformer}
整体由编码器和解码器两个比分组成。都是有多个相同的层组合而成,每层包含一个多头注意力机制和全连接前馈往回落,中间用残差连接和层归一化连接。

\paragraph{位置编码??????}
transformer和RNN不同,是并行操作整个序列,于是需要标记位置来保证顺序信息不丢失。将位置信息加和到词嵌入的特征中。

将位置转化为一个768维的向量,使用位置编码函数,找到???????

\paragraph{编码器}
原输入序列经过词嵌入和位置编码,获得输入向量。用多头注意力的h组线性曾对词嵌入表示进行变换,映射到3个人空间,获得Q,K,V。用缩放点积计算注意力权重。

\paragraph{AddNorm}
将输入向量和注意力输出向量相加,获得残差连接的结果。

对结果做层规范化。

\paragraph{多个encoder堆叠}
可以将多个编码器叠在一起,最后一个编码器的输出就是最终的编码器表示,将传递给解码器处理。

\subsection{解码器}
encoder的输出并没直接作为decoder的直接输⼊

目标序列嵌⼊和位置编码:与编码器类似,目标序列首先经过嵌⼊层的处理得到词嵌⼊,并加⼊位置编码,得到最终的输⼊向量。

要通过掩蔽多头注意力层,生成解码器自注意力矩阵,用来捕捉当前解码器状态和前面的解码器的信息。

为了防止输入序列的信息泄漏,需要将后面你的词屏蔽,使用掩码(mask)来屏蔽。将注意力矩阵和掩码相加,要掩蔽就将那个位置的掩码设为$-\inf$

接下来编码器-解码器注意⼒层
(encoder-decoder attention)将编码器
输出的信息有效地融⼊到当前解码器状
态中,以帮助解码器更好地进⾏下⼀步
预测。

在编码器-解码器注意⼒中,Q来自前
⼀个解码器层的输出,⽽K和V来自编码
器的输出。

\section{预训练模型}

预训练阶段:在无标签数据上学习通用语言表示,没有专业性。

微调阶段:针对某个特定任务进行微调,做出专业模型。

常见预训练模型:GPT和BERT。GPT单向上下文信息,BERT学习双向信息。

GPT:Generative Pre-Trained Transformer,基于Transformer的预训练生成语言模型,在大量无标签文本数据上进行预训练,学习通用表示;使用多个Transformer解码器堆叠。预测下一个词的概率

微调阶段:

为特定任务添加任务相关的头部
Head(如分类层、序列标注层
等)。

使用标注的任务数据训练进⾏微
调,优化任务相关的损失函数。

参数规模提升带来能⼒“涌现Emergent ”

原因

1. ⼤量的训练数据:⼈类积累的所有信息

2. 模型容量:充分学习数据

3. 自回归和⽆监督训练:不需要标注海量数据

4. 迁移学习和微调:适应不同任务

5. 多任务学习:提⾼泛化性



\end{document}]
