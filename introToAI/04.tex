% vim:ft=tex:
%
\documentclass[UTF8]{ctexart}
\usepackage{amsmath}
\usepackage{listings}

\lstset{
    basicstyle          =   \sffamily,          % 基本代码风格
    keywordstyle        =   \bfseries,          % 关键字风格
    commentstyle        =   \rmfamily\itshape,  % 注释的风格,斜体
    stringstyle         =   \ttfamily,  % 字符串风格
    flexiblecolumns,                % 别问为什么,加上这个
    numbers             =   left,   % 行号的位置在左边
    showspaces          =   false,  % 是否显示空格,显示了有点乱,所以不现实了
    numberstyle         =   \zihao{-5}\ttfamily,    % 行号的样式,小五号,tt等宽字体
    showstringspaces    =   false,
    captionpos          =   t,      % 这段代码的名字所呈现的位置,t指的是top上面
    frame               =   lrtb,   % 显示边框
}

\lstdefinestyle{Python}{
    language        =   Python, % 语言选Python
    basicstyle      =   \zihao{-5}\ttfamily,
    numberstyle     =   \zihao{-5}\ttfamily,
    keywordstyle    =   \color{blue},
    keywordstyle    =   [2] \color{teal},
    stringstyle     =   \color{magenta},
    commentstyle    =   \color{red}\ttfamily,
    breaklines      =   true,   % 自动换行,建议不要写太长的行
    columns         =   fixed,  % 如果不加这一句,字间距就不固定,很丑,必须加
    basewidth       =   0.5em,
}
\title{
	线性回归2
}
\author{
	Letian Lin --- \texttt{yingziyu-Lin@outlook.com}
}

\begin{document}
\maketitle
\section{广义线性回归模型}
对于任意可微函数,都可以有:$$g(y)=wX+B$$
叫做广义线性回归模型。

\section{分类问题(Logistic Regression)}
分类就是一种离散的预测。

如何应线性模型?
将标签转化为数字:$y\in \{0,1\}$

做广义线性回归模型。$y=g^{-1}(w^Tx+b)$,其中$g:=func\ which\ in\ [0,1]$

常用函数(分类器):Logistic Function(S形曲线)$y=\frac{1}{1+e^{-(w^Tx+b)}}$

使用的时候$\log\frac{y}{1-y}=w^Tx+b$
称$\log\frac{y}{1-y}$为logit

\subsection{Function}
输出映射到[0,1]之间。sigmoid函数:

$$f(x)=\sigma(\sum(w^Tx+b))$$

\subsection{Object}
使用交叉熵来做loss function.从而使预测分布和真实分布之间的K-L散度最小。

怎么做?$-\sum(\hat y^i\log P(x^i) + (1-\hat y^i)\log(1- P(x^i))$

为什么不使用Square Loss?根据导数推导出来,在远远远离正确答案的时候,梯度为0,无法正常下降,收敛非常非常慢。

\subsection{Methods}

梯度下降。略

\subsection{局限性}
只能解决线性可分的问题,不能处理异或问题。

feature transformation。
对于异或问题,可能可以找到一种变换的方法,使得其重新转化成线性可分的问题。非常困难,不一定能找到。
进行级联Logistic Regression,第一层进行feature transfomathon,第二层分类。-> 和DNN相关

\section{线性分类器解决图片分类问题}
\subsection{挑战}
视角变化、光照不同、背景杂乱、形变、类内差异,context

\subsection{流程}
数据驱动的方法。收集数据和标签的数据集,用机器学习算法训练一个分类器,在测试图片上评估分类器。

线性分类器是有参算法。

一般来说,神经网络最后一层是线性层,就是$f(W,x)=W^Tx+b$,输出是个向量,表示各类类别的可能性(score)。

参数中W的各个行,可以认为是一个滤波器,专门用来提取某一个类别的共同特征。

从几个角度解释:对图片进行多次二分类。

\section{多项式线性回归分类器}
\subsection{背景}
线性分类器产出的结果是一个值,不是概率,数据缺乏意义。

\subsection{做法}
在线性分类套个函数:soft_max函数。

soft_max函数是sigmoid的推广,是一个归一化的指数函数,所有结果的加和为1。

将所有项全部加指数,然后归一化。

用交叉熵作为Loss Function就可以了。

\section{最近邻分类器KNN}
是一个无参的方法。但还是有监督学习,是要有训练数据的。

训练就是记住所有图片,测试就是将测试数据和所有训练数据比较,找到最像的,取这个的label作为结果。

比较两个图片相似度的方法:L1 Distance:$d_1(I_1,I_2\sum_p|I_1^p-I_2^p|$

k近邻:majority vote,找一个点附近最近的点哪个label最多,就取这个label做结果,可以减少异常值的影响。

有超参数:k选取和L的选择,选择的效果和问题与数据集紧密相关,需要多次尝试。

\subsection{如何选择超参数?}
选择train最好参数:不好,过拟合;选择test最好参数:不好,不知道test;将数据分成train和validation两部分,只在train上训练,在validation数据集上面选择超参数,较好。

最好的方法:Cross Validation:将数据分成很多个部分,每次使用不同的fold作为验证集,取平均效果最好的超参数,适用于小数据集,耗时间。

最后只在测试数据上面运行一次。

\section{聚类}
无监督学习

将相似的内容聚在一起。特点:类内相似度很高,类间的相似度很低。

方法有:分割方法(k聚类等),层级方法

\subsection{K means}
流程:input->initialize->classify->re-center->classify->...->Terminate

问题:结果对初始的随机中心敏感,所以要尝试不同的随机种子,且尽量让种子相互远离;对离群值敏感
\end{document}
