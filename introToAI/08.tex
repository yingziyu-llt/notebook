% vim:ft=tex:
%
\documentclass[UTF8]{ctexart}
\usepackage{amsmath}
\usepackage{listings}

\lstset{
    basicstyle          =   \sffamily,          % 基本代码风格
    keywordstyle        =   \bfseries,          % 关键字风格
    commentstyle        =   \rmfamily\itshape,  % 注释的风格,斜体
    stringstyle         =   \ttfamily,  % 字符串风格
    flexiblecolumns,                % 别问为什么,加上这个
    numbers             =   left,   % 行号的位置在左边
    showspaces          =   false,  % 是否显示空格,显示了有点乱,所以不现实了
    numberstyle         =   \zihao{-5}\ttfamily,    % 行号的样式,小五号,tt等宽字体
    showstringspaces    =   false,
    captionpos          =   t,      % 这段代码的名字所呈现的位置,t指的是top上面
    frame               =   lrtb,   % 显示边框
}

\lstdefinestyle{Python}{
    language        =   Python, % 语言选Python
    basicstyle      =   \zihao{-5}\ttfamily,
    numberstyle     =   \zihao{-5}\ttfamily,
    keywordstyle    =   \color{blue},
    keywordstyle    =   [2] \color{teal},
    stringstyle     =   \color{magenta},
    commentstyle    =   \color{red}\ttfamily,
    breaklines      =   true,   % 自动换行,建议不要写太长的行
    columns         =   fixed,  % 如果不加这一句,字间距就不固定,很丑,必须加
    basewidth       =   0.5em,
}
\title{
  RNN
}
\author{
	Letian Lin --- \texttt{yingziyu-Lin@outlook.com}
}

\begin{document}
\maketitle

MLP和CNN都是将一个样本作为输入,输出一个结果,叫做前馈神经网络(FNN),为每个输入获得一个输出。

对于时间序列数据集(语言,视频和生物信号),不能用前馈神经网络处理。RNN专门用来处理这类问题。

\section{词的表示}
\subsection{简单方法}
one hot vector,建立一个one hot的词汇表。缺点:大词汇量出现维数灾难。所有单词都独立表示,词汇表过于稀疏。

词袋模型:统计出现了哪些词,和他们的出现的频次。缺点:维数灾难,丢失了单词的位置信息。

\subsection{词嵌入}
用一个FNN提取词汇的特征,将词汇变为一个多维浮点数的向量。

一个好的词嵌入应当是:相似语义信息应当表示相似,应该允许对语义信息进行语义操作;同时,可以用低维向量表示单词。

\subsection{习得词嵌入}
现有的算法通过阅读大量文本文档获得词嵌入表,从而获得其内部模式。

\section{Word2Vec}
CBOW:用上下文预测中间的词,多次使用相同的词嵌入表,速度快。Skip-Gram:根据中间词预测上下文。

\section{序列数据}
one to one:FNN

one to many,many to many...:RNN

\section{Vanilla RNN}
对于每一个时刻,不仅输入这个时间步下的输入$x_t$,同时还要输入上一个时刻的隐藏状态向量$h_t$,生成输出$h_{t+1}$

隐藏层$h_t=f(W_hx_t+U_hh_{t-1}+b_h)$,$h_0=0$

输出层$y_t=g(U_th_{t}+b_y)$

局限性:长期依赖问题,前期的内容很容易消失,出现遗忘。

梯度消失和梯度爆炸。

\section{长短期记忆网络LSTM}

引入门控单元来控制内容的记忆和遗忘。

门控函数:$InputVector \dot GateVector = OutputVector$,门控向量的值从0到1变化。

遗忘门:控制是否要遗忘上次的信息$f_t=sigmoid([h_{t-1},x_t]W_f+b_f)$

计算输入门:控制这次输入是否要计算$i_t$。计算信息向量$\hat{C}_i$

计算新的细胞向量:考量遗忘和计算输入门和原来的细胞向量$C_t=f_t*C_{t-1}+\hat{C}_i*i_t$

输出:考量细胞向量和输入,获得新的$h_t$

\end{document}
