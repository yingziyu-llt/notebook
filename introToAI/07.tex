% vim:ft=tex:
%
\documentclass[UTF8]{ctexart}
\usepackage{amsmath}
\usepackage{listings}

\lstset{
    basicstyle          =   \sffamily,          % 基本代码风格
    keywordstyle        =   \bfseries,          % 关键字风格
    commentstyle        =   \rmfamily\itshape,  % 注释的风格,斜体
    stringstyle         =   \ttfamily,  % 字符串风格
    flexiblecolumns,                % 别问为什么,加上这个
    numbers             =   left,   % 行号的位置在左边
    showspaces          =   false,  % 是否显示空格,显示了有点乱,所以不现实了
    numberstyle         =   \zihao{-5}\ttfamily,    % 行号的样式,小五号,tt等宽字体
    showstringspaces    =   false,
    captionpos          =   t,      % 这段代码的名字所呈现的位置,t指的是top上面
    frame               =   lrtb,   % 显示边框
}

\lstdefinestyle{Python}{
    language        =   Python, % 语言选Python
    basicstyle      =   \zihao{-5}\ttfamily,
    numberstyle     =   \zihao{-5}\ttfamily,
    keywordstyle    =   \color{blue},
    keywordstyle    =   [2] \color{teal},
    stringstyle     =   \color{magenta},
    commentstyle    =   \color{red}\ttfamily,
    breaklines      =   true,   % 自动换行,建议不要写太长的行
    columns         =   fixed,  % 如果不加这一句,字间距就不固定,很丑,必须加
    basewidth       =   0.5em,
}
\title{
	生成模型--GAN
}
\author{
	Letian Lin --- \texttt{yingziyu-Lin@outlook.com}
}

\begin{document}
\maketitle
\section{什么是生成模型}
CG和生成型:CG是基于先验知识,而生成模型是基于数据的,有一定的概率,统计、深度生成模型也有一定的先验知识。

学习的是概率分布,从概率分布中采样获得结果。

生成->密度估计->无监督表示学习。

从数据中获得潜在结构。

判别模型是找到决策边界P(Y|X),生成模型是生成数据,寻找联合分布P(Y,X)

生成模型也可以同时用来做判别。

\section{Vanilla GAN}
z(Normal Distribution)->G(generator)->$\hat{x}$/x(输入真实图片)->D(detector)->Fake/Real

$$L_D=-E_{x~p_{data}}[\log D(x)] - E_{x~p_z}[\log(1-D(G(x))]$$最大化检测正确率
$$L_G=-E_{x~p_z}[\log(D(G(X))]$$最小化检测正确率

先固定G训练D,再固定D训练G。二者都不训练到最优就换。

\section{DCGAN}
接收一个100维的随机噪声,通过多个转置卷积产生一个图像,整体架构不变。

可以调整参数调整图片生成种类。

\section{VAE}
Generater+Encoder:输入一个图片,输出一个图片,做两个之间的loss
Better GAN: encoder+generator+determinter

\section{Conditional GAN}
输入的除了一个z,还要加一个c,这个c是物体类别的label。

Condition有很多种,可以是类别、文本、文本+图像。这个是一个多模态的生成问题。

结构:D:可以将匹配的文本(由RNN-encoder给出)和图像判断为真实样本,将不匹配的文本和图像判断为伪造样本。

G:可以根据文本(由RNN-encoder给出)生成图像,训练欺骗D。

\section{BiGAN}
encoder+generator+determinter

\section{CoGAN}
学习两个语义相近的领域的联合分布,做到从一个领域图像映射到另一个领域。将输入放入两个Vanilla GAN,但G的前几层还有D的最后几层两个网络共享。训练就可。

只能生成一对对图像,但不能给出一张图,获得对应的另一张图。

\section{CycleGAN}
修改BiGAN,改为$x->G->\hat{x}->D$,重复两次。D读取两个$\hat{x}$。

Cycle Loss$a->G_{AToB}->\hat{b}->G{BToA}->\hat{a}$,loss为$a和\hat{a}$之间的loss。

\end{document}
