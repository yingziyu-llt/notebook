% vim:ft=tex:
%
\documentclass[UTF8]{ctexart}
\usepackage{amsmath}
\usepackage{listings}

\lstset{
    basicstyle          =   \sffamily,          % 基本代码风格
    keywordstyle        =   \bfseries,          % 关键字风格
    commentstyle        =   \rmfamily\itshape,  % 注释的风格,斜体
    stringstyle         =   \ttfamily,  % 字符串风格
    flexiblecolumns,                % 别问为什么,加上这个
    numbers             =   left,   % 行号的位置在左边
    showspaces          =   false,  % 是否显示空格,显示了有点乱,所以不现实了
    numberstyle         =   \zihao{-5}\ttfamily,    % 行号的样式,小五号,tt等宽字体
    showstringspaces    =   false,
    captionpos          =   t,      % 这段代码的名字所呈现的位置,t指的是top上面
    frame               =   lrtb,   % 显示边框
}

\lstdefinestyle{Python}{
    language        =   Python, % 语言选Python
    basicstyle      =   \zihao{-5}\ttfamily,
    numberstyle     =   \zihao{-5}\ttfamily,
    keywordstyle    =   \color{blue},
    keywordstyle    =   [2] \color{teal},
    stringstyle     =   \color{magenta},
    commentstyle    =   \color{red}\ttfamily,
    breaklines      =   true,   % 自动换行,建议不要写太长的行
    columns         =   fixed,  % 如果不加这一句,字间距就不固定,很丑,必须加
    basewidth       =   0.5em,
}
\title{
	卷积神经网络
}
\author{
	Letian Lin --- \texttt{yingziyu-Lin@outlook.com}
}

\begin{document}
\maketitle
\section{目的}
对于高维数据,不能对于每个像素进行建模,就是建起来了也放不进内存,所以不能直接建模。

同时,人眼也不是一个全连接层,对于图片也有平移不变性,所以不能用全连接层建立模型。需要用filter卷积后存入。

\section{卷积}
特征:当滤波器的局部路径在某个位置很像,那么卷积操作输出序列的某个点就会很高。

参数:filter,padding,strides

感受野:特征图上某个点在原始图片上的范围大小叫做感受野,一般来说,感受野越大越好,但会带来参数数量的迅速变大,最后导致极难调参。

对于分割类的模型,使用小的感受野可能效果会更好,可以更加精确的分割内容。

\section{优化}
\subsection{空洞卷积}
目的还是增加感受野,同时避免增加参数。在filter上面插入一些0.dilation rate:插0的比例。

\subsection{pooling algorithm(池化算法)}
目的:增强平移不变性。

空间金字塔池化:对原数据多次pooling,变成很多张特征图,拼接在一起,再去训练。

\subsection{Hieratchial Representation Learning(分层表示学习)}
低层次的特征是很low leval的特征,是细粒度的特征,而高层次的特征是粗粒度的特征,会丢失特征。

\subsection{Convolutional Architectures(卷积神经⽹络结构)}
AlexNet:第一个比较成功的图像分类网络。

VGG:用小filter,大量连续的卷积层,产生大量非线性的内容。

ResNet:增加了一个残差块。

SqueezeNet, MobileNet, ShuffleNet:轻量化卷积网络。

\subsection{反卷积}
在原来的输入上插入了一些0,输出的伪书反而变大了。

\section{应用}
图片分类(略)
\subsection{目标检测}
评价:交并比(IoU):选择预测结果和ground truth两个框,看两个框交集和并集的比,越大越好。

平均准确度(AP)$Precision = \frac{TP}{TP + FP}$
$Recall = \frac{TP}{TP + FN}$,可以做出Precision-Recall曲线,曲线下的面积就是AP。对所有类别下的AP求均值叫做mAP.

\paragraph{R-CNN}
在输入图片上生成好多个框(Region Proposals)(使用选择性搜索方法),将框调整成相同的大小,然后跑CNN分类。
\subparagraph{NMS}对每个物体都会有好多个分类框,对这些分类框选择打分最高的保留,其他丢弃。
\subparagraph{局限}搜索速度慢。resize导致宽度和高度变化,影响分类准确性。将每个区域输入VGG里面很慢。非端到端的。

\paragraph{SPP Net}解决了第二和第三个缺陷。使用空间金字塔池化,在特征图上修改大小。在全局上用CNN,然后在选候选区域。
\subparagraph{局限}依然搜索慢,仍然不是端到端。

\paragraph{fast R-CNN}不用空间金字塔池化改用ROI,用神经网络做分类器。

\paragraph{Faster R-CNN}使用区域提议网络(RPN)代替搜索,真正端到端。

\paragraph{YOLO}
只看一次而非像先前一样做区域提议。

把目标检测转化为回归问题,使用完全卷积网络(FCN)输出类别标签和位置信息。

\paragraph{SSD}
和YOLO类似,也是一个单阶段的算法。

\subsection{图像分割}
Semantic Segmentation:仅按类别分割,不区分个体。Instance Segmentation:区分个体。

本质就是像素级的分类问题。可以用多通道的提议
\subsubsection{Semantic Segmentation}

FCN:全卷积神经网络。只适用卷积和池化操作,可以接受任意大小的输入图像。encoder适用池化和步幅卷积进行下采样,decoder适用转置卷积进行上采样。

由于卷积会使得特征越来越小,于是使用skip connection

\paragraph{loss function}:像素级的交叉熵:小物体性能差

Dice coefficient:$Dice = \frac{|A\and B|}{|A| + |B|}$

\subsubsection{Instance Segmentation}

Instance Segmentation = Object Classfication + Object Detection + Semantic Segmentation

\end{document}
