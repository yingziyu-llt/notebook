% vim:ft=tex:
%
\documentclass[UTF8]{ctexart}
\usepackage{graphicx} % 插图片
\usepackage{amsmath}  % 数学公式拓展
\usepackage{float}    % 浮动体控制
\usepackage{listings} % 插入代码,不过据说 minted 的效果更好一些
\title{
	线性方程组
}
\author{
	Letian Lin --- \texttt{yingziyu-Lin@outlook.com}
}
\date{February,19 2024}
\begin{document}
\maketitle
\section{线性方程组的概念}
\paragraph{线性方程:}线性方程就是$n$元$1$次方程,要求没有二次项,没有交叉项,也没有其他函数。

例子:$a_1x_1+a_2x_2+\dots+a_nx_n=b$
\paragraph{线性方程组:}由\textbf{线性方程}组成的方程组,如:
$$\begin{cases}
	a_{11}x_1+a_{12}x_2+\dots+a_{1n}xn=b_1\\
	a_{21}x_1+a_{22}x_2+\dots+a_{12}xn=b_2\\
	...
	a_{m1}x_1+a_{m2}x_2+\dots+a_{mn}=b_m
\end{cases}$$
我们将$x_1,x_2\dots x_n$称为变元,$a_{11},a_{12}\dots a_{1n} \dots a_{mn}$称为系数,$b_1,b_2,\dots b_m$称为常数。

\paragraph{解}若$c_1,c_2,\dots c_m$带入方程$x_1\dots x_n$,使得每个等式都成立,称
$\begin{pmatrix}
	c_1\\
	c_2\\
\vdots\\
	c_m
\end{pmatrix}$
为一组解,所有解构成的集合叫做解集

\section{高斯-约当消元法}
\paragraph{基本思路}从左向右,从上向下,依次消元,做出阶梯矩阵后从下向上消元,将增广矩阵约化为简化阶梯矩阵。
\paragraph{初等变换}针对方程组有三种初等变换,经过初等变换,方程组的解不变。
\begin{enumerate}
	\item 把某方程的某倍数加到另一方程上\\
	\item 互换两方程的位置\\
	\item 用非零数乘以某方程
\end{enumerate}

\paragraph{矩阵}
把$s \times m$个数排成$s$行$m$列,就称为$s\times m$ 矩阵。元素均为0的称做0矩阵,记作\textbf{0},$s=m$的矩阵称为方阵。

\paragraph{矩阵的初等行变换}有三种初等行变换。
\begin{enumerate}
	\item 把某行的某倍数加到另一方程上\\
	\item 互换两行的位置\\
	\item 用非零数乘以某行
\end{enumerate}

\paragraph{阶梯形矩阵和简化阶梯形矩阵}

阶梯形矩阵要求:1.元素全为$0$的行(零行)在矩阵最下方     2.左边第一个非零元素(主元)的列指标随行指标\textbf{严格}递增

简化阶梯形矩阵要求:1.是阶梯形矩阵    2.每个主元都是1   3.每个主元所在列的其余元素都是0

简化阶梯形矩阵例
$$\begin{bmatrix}
	1&0&0&0&3\\
	0&0&1&0&2\\
	0&0&0&1&1\\
	0&0&0&0&0\\
	0&0&0&0&0
\end{bmatrix}$$

\end{document}
