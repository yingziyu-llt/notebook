% vim:ft=tex:
%
\documentclass[UTF8]{ctexart}
\usepackage{amsmath}
\usepackage{listings}

\lstset{
    basicstyle          =   \sffamily,          % 基本代码风格
    keywordstyle        =   \bfseries,          % 关键字风格
    commentstyle        =   \rmfamily\itshape,  % 注释的风格,斜体
    stringstyle         =   \ttfamily,  % 字符串风格
    flexiblecolumns,                % 别问为什么,加上这个
    numbers             =   left,   % 行号的位置在左边
    showspaces          =   false,  % 是否显示空格,显示了有点乱,所以不现实了
    numberstyle         =   \zihao{-5}\ttfamily,    % 行号的样式,小五号,tt等宽字体
    showstringspaces    =   false,
    captionpos          =   t,      % 这段代码的名字所呈现的位置,t指的是top上面
    frame               =   lrtb,   % 显示边框
}

\lstdefinestyle{Python}{
    language        =   Python, % 语言选Python
    basicstyle      =   \zihao{-5}\ttfamily,
    numberstyle     =   \zihao{-5}\ttfamily,
    keywordstyle    =   \color{blue},
    keywordstyle    =   [2] \color{teal},
    stringstyle     =   \color{magenta},
    commentstyle    =   \color{red}\ttfamily,
    breaklines      =   true,   % 自动换行,建议不要写太长的行
    columns         =   fixed,  % 如果不加这一句,字间距就不固定,很丑,必须加
    basewidth       =   0.5em,
}

\title{
	从C到C++
}
\author{
	Letian Lin --- \texttt{yingziyu-Lin@outlook.com}
}

\date{Feb 23,2024}

\begin{document}
\maketitle

\section{引用的概念和应用}
定义引用:类型名\&引用名=某变量名;

只是别名,修改引用的同时会修改原值。

定义的时候必须要初始化,不能修改引用的对象,不可引用常量和表达式。

e.g. int \&r = n;

\begin{lstlisting}[language=C++]
double a = 4,b = 5;
double &r1 = a;
double &r2 = r1;
r2 = 10;
cout << a << endl; //10
r1 = b;
cout << a << endl; //5;
\end{lstlisting}

引用还可以作为函数返回类型,只需要return相关的变量,就可以ruturn对应变量的引用。

可以用const修饰,不能修改用const修饰的引用。

\begin{lstlisting}[language=c++]
int n = 4;
const int &a = n;
a = 10;//编译错误
n = 10;//正常
\end{lstlisting}

const T\&的引用是不可以用来初始化T \&的内容的,二者的类型其实不相同,只能在强制类型转化后进行初始化。

\section{const 关键字}

\subsection{定义常量}

\begin{lstlisting}[language=C++]
const int MAXVALUE = 10000000;//建议常量使用全大写
const string NAME = "yingziyu";
\end{lstlisting}

\subsection{定义常量指针}
常量指针指向的内容只读,但其自己可以被修改指向别的内容。

\begin{lstlisting}[language=C++]
const int *p = n;
* p = 5 //CE
n = 5;  //OK
p = &m; //OK
\end{lstlisting}

不可将常量指针赋值给普通指针。

常量指针还可以避免函数内部将参数指针的所指的内容修改。
\begin{lstlisting}[language=C++]
void myPrint(const char *s){
	strcpy(s,"this");    //CE
	printf("%s",s);      //OK
}
\end{lstlisting}



\section{动态内存分配}

\subsection{分配一个变量}
用new申请内存,delete释放内存
\begin{lstlisting}[language=C++]
int *pn = new int;
*pn = 5;
\end{lstlisting}
\subsection{分配一个数组}
P = new T[N];

T是任意类型名,N是一个整数。P指向数组的第一位。

\subsection{delete}
销毁用new分配出来的动态空间。

只能delete new出来的空间。

\begin{lstlisting}[language=C++]
int *p=new int;
*p = 5;
delete p;
delete p;//异常,不可多次delete
\end{lstlisting}

delete数组,需要加[]
\begin{lstlisting}[language=C++]
int *p = new int[100];
p[0] = 1;
delete[] p;coi
\end{lstlisting}

\section{内联函数,函数重载,函数缺省参数}

\subsection{内联函数}

函数调用有时间开销,若函数本身很简单,而且多次被执行,就会导致调用时间开销很大。

内联函数在被编译器处理的时候,是将内联函数代码直接插入调用语句的位置,而不会调用。

\begin{lstlisting}[language=C++]
inline int Max(int a,int b)
{
	if(a > b) return a;
	return b;
}
\end{lstlisting}

但内联函数会使得代码长度变长,可执行文件大小变大。

\subsection{函数重载}

函数名字相同,参数类型或个数不同,就可以重载函数。

需要保证没有二义性。

\subsection{缺省}

定义函数的时候可以让\textbf{最右边的连续若干个}参数有缺省,调用时不写参数就是缺省
\begin{lstlisting}
void func(int x1,int x2 = 1,int x3 = 2)
{}
func(10);//==func(10,1,2);
func(10,8);//==func(10,8,2);
func(10,,8);//ERROR
\end{lstlisting}

\end{document}
