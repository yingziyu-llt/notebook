% vim:ft=tex:
%
\documentclass[UTF8]{ctexart}
\usepackage{amsmath}
\usepackage{listings}

\lstset{
    basicstyle          =   \sffamily,          % 基本代码风格
    keywordstyle        =   \bfseries,          % 关键字风格
    commentstyle        =   \rmfamily\itshape,  % 注释的风格,斜体
    stringstyle         =   \ttfamily,  % 字符串风格
    flexiblecolumns,                % 别问为什么,加上这个
    numbers             =   left,   % 行号的位置在左边
    showspaces          =   false,  % 是否显示空格,显示了有点乱,所以不现实了
    numberstyle         =   \zihao{-5}\ttfamily,    % 行号的样式,小五号,tt等宽字体
    showstringspaces    =   false,
    captionpos          =   t,      % 这段代码的名字所呈现的位置,t指的是top上面
    frame               =   lrtb,   % 显示边框
}

\lstdefinestyle{Python}{
    language        =   Python, % 语言选Python
    basicstyle      =   \zihao{-5}\ttfamily,
    numberstyle     =   \zihao{-5}\ttfamily,
    keywordstyle    =   \color{blue},
    keywordstyle    =   [2] \color{teal},
    stringstyle     =   \color{magenta},
    commentstyle    =   \color{red}\ttfamily,
    breaklines      =   true,   % 自动换行,建议不要写太长的行
    columns         =   fixed,  % 如果不加这一句,字间距就不固定,很丑,必须加
    basewidth       =   0.5em,
}
\title{
	cpp 11
}
\author{
	Letian Lin --- \texttt{yingziyu-Lin@outlook.com}
}

\begin{document}
\maketitle
\paragraph{统一初始化方法}使用大括号左作为数组或者容器作为统一初始化方法。

可以防止缩窄,即禁止将数值放入无法放置它的变量。

\paragraph{成员变量默认初始值}
可以在定义类的时候直接给成员变量设初值。

\paragraph{auto\&decltype}定义变量的时候使用auto可以自动推断类型,用auto做返回值类型需要加->decltype()。decltype:推导表达式的类型。

\paragraph{shared_ptr}用shared_ptr可以让其托管一个new运算符返回的指针,在没有指针指向这个位置的时候自动delete。
不要混用智能指针和普通指针,否则有crash的可能性(多次delete)。

\paragraph{range for}for(auto i : a)

\paragraph{右值引用和move}一般来说, 不能取地址的表达式, 就是右值。主要目的是提高程序运行的效率, 减少需要进行深拷贝的对象进行深拷贝的次数。

\paragraph{unordered_map}就是个hash表。unordered_map<key,val> ump;ump.insert(pair<key,val>);ump[key] = val;ump.find(key)->unordered_map::iterator;

\paragraph{regex}regex reg("your_regex_here");regex_match("your_string_here",reg)->bool;

\paragraph{lambda}定义:auto fn=[修饰表](参数表){变量}->返回值;调用:fn()

修饰表:=传值,\&传引用

function<返回值(参数表)> func;可以做lambda或者自由函数。
\end{document}
