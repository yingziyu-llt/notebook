% vim:ft=tex:
%
\documentclass[UTF8]{ctexart}
\usepackage{amsmath}
\usepackage{listings}

\lstset{
    basicstyle          =   \sffamily,          % 基本代码风格
    keywordstyle        =   \bfseries,          % 关键字风格
    commentstyle        =   \rmfamily\itshape,  % 注释的风格,斜体
    stringstyle         =   \ttfamily,  % 字符串风格
    flexiblecolumns,                % 别问为什么,加上这个
    numbers             =   left,   % 行号的位置在左边
    showspaces          =   false,  % 是否显示空格,显示了有点乱,所以不现实了
    numberstyle         =   \zihao{-5}\ttfamily,    % 行号的样式,小五号,tt等宽字体
    showstringspaces    =   false,
    captionpos          =   t,      % 这段代码的名字所呈现的位置,t指的是top上面
    frame               =   lrtb,   % 显示边框
}

\lstdefinestyle{Python}{
    language        =   Python, % 语言选Python
    basicstyle      =   \zihao{-5}\ttfamily,
    numberstyle     =   \zihao{-5}\ttfamily,
    keywordstyle    =   \color{blue},
    keywordstyle    =   [2] \color{teal},
    stringstyle     =   \color{magenta},
    commentstyle    =   \color{red}\ttfamily,
    breaklines      =   true,   % 自动换行,建议不要写太长的行
    columns         =   fixed,  % 如果不加这一句,字间距就不固定,很丑,必须加
    basewidth       =   0.5em,
}
\title{
	template
}
\author{
	Letian Lin --- \texttt{yingziyu-Lin@outlook.com}
}

\begin{document}
\maketitle
\section{函数模板}
格式:template<class T1,class T2...>

返回值类型 参数名(参数表)\{
\}

主要用在对不同的对象,用很类似的函数处理。

在每一条语句里面,某一个类型标识符只能代表一个类型,如T不能同时被当成int和float。

先找参数完全匹配的函数,在找一个参数完全匹配的模板,再找参数经过自动能够类型转换之后的可以匹配的函数,要都不行就报错。

\section{泛型程序设计}
算法实现的时候不指定具体要操作的数据类型,算法实现一遍,适用于多种数据结构,可以减少重复代码的编写。

\section{类模板}
和函数模板很像。一类“类”除了内部成员变量的类型不一样之外,其他的都一样,就可以用类模板来处理。

类模板的成员函数在类模板外定义的时候需要:
\begin{lstlisting}
template<class T1,class T2>
bool Pair<T1,T2>::operatoer(const Pair<T1,T2> &p)const{
	return key < p.key;
}
\end{lstlisting}

模板类:被实例化后的类模板,由编译器自动生成。

类模板里面的成员函数也可以做函数模板,和普通的函数模板的格式一样,但需要注意内外的模板名称不应当一致。同时,该成员函数只有被调用的时候才会被实例化。

类模板中的属性参数表可以出现非类型参数,用来说明类的一些属性,比如:
\begin{lstlisting}
template <class T,int size>
class Array{
	T array[size];
};

Array<double,40> a2;
\end{lstlisting}

派生:类模板和普通类可以相互派生。
\begin{lstlisting}
template <class T1,class T2>
class A{};

template<class T1,class T2>//类模板从类模板派生
class B:public A<T1,T2>{};

template<class T>
class C:public B<T,T>{};

template<class T>//类模板从模板类派生
class D:public A<int,double>{};

class E{};

template<class T>//类模板从普通类派生
class F:public E{};

class G:public A<int,int>{};//普通类从类模板派生
\end{document}
