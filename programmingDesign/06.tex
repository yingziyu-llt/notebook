% vim:ft=tex:
%
\documentclass[UTF8]{ctexart}
\usepackage{amsmath}
\usepackage{listings}

\lstset{
    basicstyle          =   \sffamily,          % 基本代码风格
    keywordstyle        =   \bfseries,          % 关键字风格
    commentstyle        =   \rmfamily\itshape,  % 注释的风格,斜体
    stringstyle         =   \ttfamily,  % 字符串风格
    flexiblecolumns,                % 别问为什么,加上这个
    numbers             =   left,   % 行号的位置在左边
    showspaces          =   false,  % 是否显示空格,显示了有点乱,所以不现实了
    numberstyle         =   \zihao{-5}\ttfamily,    % 行号的样式,小五号,tt等宽字体
    showstringspaces    =   false,
    captionpos          =   t,      % 这段代码的名字所呈现的位置,t指的是top上面
    frame               =   lrtb,   % 显示边框
}

\lstdefinestyle{Python}{
    language        =   Python, % 语言选Python
    basicstyle      =   \zihao{-5}\ttfamily,
    numberstyle     =   \zihao{-5}\ttfamily,
    keywordstyle    =   \color{blue},
    keywordstyle    =   [2] \color{teal},
    stringstyle     =   \color{magenta},
    commentstyle    =   \color{red}\ttfamily,
    breaklines      =   true,   % 自动换行,建议不要写太长的行
    columns         =   fixed,  % 如果不加这一句,字间距就不固定,很丑,必须加
    basewidth       =   0.5em,
}
\title{
	继承和派生
}
\author{
	Letian Lin --- \texttt{yingziyu-Lin@outlook.com}
}

\begin{document}
\maketitle
\section{为什么要有继承}
很多对象都有相同的属性,使用继承,避免重复编写,加快编码速率。

举例:学生们都有很多共同的属性和方法,比如姓名、年纪、学号、是否优秀、是否留级等。
而学生也有很多不同点,如本科生有是否保研、院系,研究生有导师、论文等。
为了避免重复编码,可以先定义一个基类“学生”,从此派生出本科生、研究生等。

\section{实现}
语法:class 派生类名:派生方式 基类名

派生类有基类的所有成员变量,但其成员函数不可以访问基类的private成员。

\begin{lstlisting}[language=C++]
class CStudent{
public:
	char name[100];
	void setname(char* s);

private:
	int privateNumber;
	protected:
	int protectedNumber;
};
class CGraduateStudent : public CStudent {
private:
	int nDepartment;
	char szSupervisorName[20];

public:
	int CountSalary() { ... }
};
\end{lstlisting}
\section{继承和复合}
继承是“是”关系,派生类B也是一个基类A,也就是A是B的一个上位概念,比如Man和Woman都应该是Human的派生。
复合是有关系,是类B中有A类型的元素,比如Man里面复合了Brain元素。

派生类里面可以有和基类相同名称的元素和函数,叫做覆盖,调用时默认值是覆盖的元素。若在派生类里面访问基类的某被覆盖的元素,需要基类名称::

\section{访问权限}
private标识符:只能在基类的方法和友元访问

public标识符:基类成员和友元,派生类的成员和友元,其他函数

protected标识符:基类的成员函数

基类的友员函数

派生类的成员函数可以访问当前对象的基类的保护成员,
this指向的对象,
也可以访问其他同类对象的基类的保护成员,
派生类的友元函数也可以访问基类的保护成员

继承类型:以什么类型继承,基类成员最高访问权限就是这个类型,高于这个的就设置为这个。

\section{构造函数和析构函数}
创建派生类的对象时,需要调用基类的构造函数。显式调用:\text{derived::derived(arg_derived-list):base(arg_base-list)}

析构函数调用时,自动调用自己的析构函数,然后自动调用基类析构函数。

\section{杂项}
\subsection{public继承}
派生类的对象可以赋值给基类对象;
派生类对象可以初始化基类引用;
派生类对象的地址可以赋值给基类指针

\subsection{直接基类和间接基类}
直接相连就是直接基类,中间有类间隔就是简介基类的关系。

声明派生的时候只需列出直接基类就可以了。

\subsection{多继承}
一个类从多个类继承而来,存在二义性。
\end{document}
