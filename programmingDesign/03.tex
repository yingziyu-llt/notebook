% vim:ft=tex:
%
\documentclass[UTF8]{ctexart}
\usepackage{amsmath}
\usepackage{listings}

\usepackage{xcolor}
\lstset { %
    language=C++,
    backgroundcolor=\color{black!5}, % set backgroundcolor
    basicstyle=\footnotesize,% basic font setting
}
\lstset{
    basicstyle          =   \sffamily,          % 基本代码风格
    keywordstyle        =   \bfseries,          % 关键字风格
    commentstyle        =   \rmfamily\itshape,  % 注释的风格,斜体
    stringstyle         =   \ttfamily,  % 字符串风格
    flexiblecolumns,                % 别问为什么,加上这个
    numbers             =   left,   % 行号的位置在左边
    showspaces          =   false,  % 是否显示空格,显示了有点乱,所以不现实了
    numberstyle         =   \zihao{-5}\ttfamily,    % 行号的样式,小五号,tt等宽字体
    showstringspaces    =   false,
    captionpos          =   t,      % 这段代码的名字所呈现的位置,t指的是top上面
    frame               =   lrtb,   % 显示边框
}

\lstdefinestyle{Python}{
    language        =   Python, % 语言选Python
    basicstyle      =   \zihao{-5}\ttfamily,
    numberstyle     =   \zihao{-5}\ttfamily,
    keywordstyle    =   \color{blue},
    keywordstyle    =   [2] \color{teal},
    stringstyle     =   \color{magenta},
    commentstyle    =   \color{red}\ttfamily,
    breaklines      =   true,   % 自动换行,建议不要写太长的行
    columns         =   fixed,  % 如果不加这一句,字间距就不固定,很丑,必须加
    basewidth       =   0.5em,
}
\title{
	面向对象编程
}
\author{
	Letian Lin --- \texttt{yingziyu-Lin@outlook.com}
}
\date{\today}
\begin{document}
\maketitle
\section{基本知识}

\paragraph{抽象}将客观食物共同特点归纳出来$\rightarrow$形成数据结构$\rightarrow$将行为归纳出来$\rightarrow$编写方法$\rightarrow$形成一个class

\paragraph{封装}将数据结构和函数捆绑在一起,形成类

面向对象的程序=类+类+类+$\dots$ +类

类$\rightarrow$ 成员变量+成员函数

\begin{lstlisting}[language=C++]
class CRectangle//用C作为第一个字母
{
	public:
		int w,h;
		int area(){
			return w * h;
		}
};
\end{lstlisting}

\paragraph{访问成员变量}:变量.成员;指针->成员;引用名.成员名

\paragraph{成员函数定义}:可以将函数体和类的定义分开写。要用类名::函数名来定义
\begin{lstlisting}[language=C++]
class CRectangle
{
	public:
		int w, h;
		int Area(); //成员函数在此声明
		int Perimeter();
		void Init( int w_, int h_ );
};
int CRectangle::Area() {
	return w * h;
}
int CRectangle::Perimeter() {
	return 2 * ( w + h);
}
void CRectangle::Init( int w_, int h_ ){
	w = w_; h = h_;
}
\end{lstlisting}

\paragraph{访问权限}private:私有成员,只能在class内部访问。

public:公有成员,可以在外部访问。

protected:保护成员

缺省当作private

在类的成员函数内部,可以访问当前对象的全部属性和函数,同类其他对象的全部属性和函数。

\section{构造函数}定义了构造函数,就不会默认生成无参数构造函数了。

对象生成,就不能再次被构造,在生成的的时候默认调用构造函数。

要求:构造函数必须无返回值,返回值类型也不可为void

若调用构造函数参数不足,缺省为0

对于数组,可以认为是依次使用构造函数。

拷贝构造函数
\begin{lstlisting}[language=C++]
Complex c1;//调用缺省构造函数
Complex c2(c1);//将c2初始化为c1
\end{lstlisting}

自己定义拷贝构造函数(下述代码在class内)
\begin{lstlisting}[language=C++]
Complex(Complex &e){
	real = e.real;
	imag = e.real;
	cout << ""COPY";
}
\end{lstlisting}

不允许形如X::X(X a)的构造函数

若某函数的一个参数是class A的对象,那么该函数调用时,会调用拷贝构造函数

若返回值是A的对象,那么函数返回的时候,A的拷贝构造函数会被调用

"="赋值不会导致复制构造函数被调用

参数表使用T& obj做参数,可以减小开销,还可以使用const修饰。

类型转化构造函数:只有一个参数的构造函数,运行时会自动建立一个无名变量,调用该构造函数,然后赋值。使用explicit关键字修饰,只可以显式调用。

\section{析构函数}和构造函数恰好相反。名字要与类名相同,在前面加~,没有参数和返回值,每个类只有一个析构函数。在对象消亡时被调用,做对象消亡前做好善后工作。

在定义时没有写析构函数,就会自动生成空的析构函数。

对象数组在生命周期结束的时候,其各个元素的析构函数都会被调用。

delete函数也会造成析构函数的调用,在函数调用返回后也会被调用。

\end{document}

