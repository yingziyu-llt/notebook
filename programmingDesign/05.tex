% vim:ft=tex:
%
\documentclass[UTF8]{ctexart}
\usepackage{amsmath}
\usepackage{listings}

\lstset{
    basicstyle          =   \sffamily,          % 基本代码风格
    keywordstyle        =   \bfseries,          % 关键字风格
    commentstyle        =   \rmfamily\itshape,  % 注释的风格,斜体
    stringstyle         =   \ttfamily,  % 字符串风格
    flexiblecolumns,                % 别问为什么,加上这个
    numbers             =   left,   % 行号的位置在左边
    showspaces          =   false,  % 是否显示空格,显示了有点乱,所以不现实了
    numberstyle         =   \zihao{-5}\ttfamily,    % 行号的样式,小五号,tt等宽字体
    showstringspaces    =   false,
    captionpos          =   t,      % 这段代码的名字所呈现的位置,t指的是top上面
    frame               =   lrtb,   % 显示边框
}

\lstdefinestyle{Python}{
    language        =   Python, % 语言选Python
    basicstyle      =   \zihao{-5}\ttfamily,
    numberstyle     =   \zihao{-5}\ttfamily,
    keywordstyle    =   \color{blue},
    keywordstyle    =   [2] \color{teal},
    stringstyle     =   \color{magenta},
    commentstyle    =   \color{red}\ttfamily,
    breaklines      =   true,   % 自动换行,建议不要写太长的行
    columns         =   fixed,  % 如果不加这一句,字间距就不固定,很丑,必须加
    basewidth       =   0.5em,
}
\title{
	运算符的重载
}
\author{
	Letian Lin --- \texttt{yingziyu-Lin@outlook.com}
}

\begin{document}
\maketitle
\section{含义}
将cpp默认的运算符进行重新的定义,使得其可以在非默认类型下正常运行,叫做重载。

可以让程序更加直观。

\section{重载}
\subsection{常规运算符的重载}
格式:
\begin{lstlisting}[language=C++]
Class Complex
{
public:
	double real,image;
	Complex(double r = 0,double i = 0):real(r),image(i){}
	Complex operator+ (const Complex& e)
	{
		Complex newComplex(real,image);
		newComplex.real += e.real;
		newComplex.image += e.image;
		return newComplex;
	}
}
Complex operator- (const Complex &a,const Complex &e)
{
	Complex newComplex(a.real,a.image);
	newComplex.real -= e.real;
	newComplex.image -= e.image;
	return newComplex;

}
\end{lstlisting}
\subsection{=的重载}
只能重载为成员函数。

两边的类型可以不匹配。

举例:
\begin{lstlisting}[language=C++]
String & operator= (const char* str);//类里面定义
String &operator= (const String &str);
String& String::operator= (const char* str)
{
	del []s;
	s = new char[strlen(str) + 1];
	strcpy(s,str);
	return *this;
}
String& String::operator= (const String &str)
{
	if(this == &str) return *this;
	delete [] s;
	s = new char[strlne(str.s) + 1];
	strcpy(s,str.s);
	return *this;
}
\end{lstlisting}

为什么要返回引用?和cpp的原生=运算符保持一致,便于处理a=b=c的式子。

\subsection{重载为友元}
一般来说是重载运算符作为成员函数。

要是不能满足要求,可能需要重载为普通函数。e.g. 3+a(用成员函数会报错,3没有operator+(T a))

为了访问私有变量,使用友元重载为友元普通函数。

只需要在类里声明 friend\ T\ operator+(int\ a,T\ x);即可。

\subsection{流插入运算符的重载}忽略

\subsection{[]的重载}
返回值应当是int\&之类的。例:
\begin{lstlisting}[language=C++]
class Array{
public:
	Array(int n = 10) : size(n) {
		ptr = new int[n];
	}
	~Array() {
		delete [] ptr;
	}
	int & operator[](int subscript){
		return ptr[subscript];
	}
private:
	int size;
	int *ptr;
};
\end{lstlisting}

\subsection{类型转换运算符的重载}
必须为成员函数。参数表必须为空。不指定返回值类型。
\begin{lstlisting}[language=C++]
class ...
{
	...
public:
	int a;
	operator int(){
		return a;
	}
}
\end{lstlisting}

\subsection{自增和自减运算符}
前置运算符作为一元运算符重载。后置运算符作为二元运算符重载,带一个没有用处的参数
\end{document}
