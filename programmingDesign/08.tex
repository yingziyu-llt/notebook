% vim:ft=tex:
%
\documentclass[UTF8]{ctexart}
\usepackage{amsmath}
\usepackage{listings}

\lstset{
    basicstyle          =   \sffamily,          % 基本代码风格
    keywordstyle        =   \bfseries,          % 关键字风格
    commentstyle        =   \rmfamily\itshape,  % 注释的风格,斜体
    stringstyle         =   \ttfamily,  % 字符串风格
    flexiblecolumns,                % 别问为什么,加上这个
    numbers             =   left,   % 行号的位置在左边
    showspaces          =   false,  % 是否显示空格,显示了有点乱,所以不现实了
    numberstyle         =   \zihao{-5}\ttfamily,    % 行号的样式,小五号,tt等宽字体
    showstringspaces    =   false,
    captionpos          =   t,      % 这段代码的名字所呈现的位置,t指的是top上面
    frame               =   lrtb,   % 显示边框
}

\lstdefinestyle{Python}{
    language        =   Python, % 语言选Python
    basicstyle      =   \zihao{-5}\ttfamily,
    numberstyle     =   \zihao{-5}\ttfamily,
    keywordstyle    =   \color{blue},
    keywordstyle    =   [2] \color{teal},
    stringstyle     =   \color{magenta},
    commentstyle    =   \color{red}\ttfamily,
    breaklines      =   true,   % 自动换行,建议不要写太长的行
    columns         =   fixed,  % 如果不加这一句,字间距就不固定,很丑,必须加
    basewidth       =   0.5em,
}
\title{
	IO和文件操作
}
\author{
	Letian Lin --- \texttt{yingziyu-Lin@outlook.com}
}

\begin{document}
\maketitle
\section{输入输出相关的类}
istream:输入流,cin就是其中的对象  ostream:输出流,cout是其中对象。

ifstream文件读取类,ofstream:文件写入流,fstream:文件读取+写入流

输入流对象:

cin:与标准输入设备相连,用于从键盘读取数据,也可被重定向为从文件读取数据。

cout:与标注输出设备相连,可以被重定向。

cerr:与标准错误输出设备相连,直接输出

clog:与标准错误输出设备相连,写入缓冲区

istream \&getline(char *buf,int bufSize):从输入流中读取bufSize,或直到读到'\\n'为止,可以加第三个参数作为分隔符

若超出bufSize,本次读取可行,下次读取会出错。

putback(char c):把c放回输入流,ignore2
\section{流操纵算子}
(在iomanip里面)

整数流的基数,浮点数精度,设置域宽,自行定义

fixed/scientific:定点表示/科学记数法表示

setprecision在浮点上是有效数据,在定点和科学记数法上是小数点位数
\section{数据的层次}
建立顺序文件:(fstream)

ofstream outFilr("data",ios::out | ios::binary)

ios::out 输出到文件,覆盖;ios::app:附加模式;ios::ate:任意位置写;ios::binary:二进制模式

还可以:ofstream fout;fout.open("data",ios::out | ios::binary)

ifstream类似,使用和cin很类似。

二进制文件的读写:读InFile.read((char *)\&s,sizeof(s));写:OutFile.write((char *)\&s,sizeof(s));

fstream有seekp成员函数,是移动文件指针的操作。

二进制打开只在windows下和普通打开文件的方法不同,主要是\\r\\n的问题。
\end{document}
