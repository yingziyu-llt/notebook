% vim:ft=tex:
%
\documentclass[UTF8]{ctexart}
\usepackage{amsmath}
\usepackage{listings}

\lstset{
    basicstyle          =   \sffamily,          % 基本代码风格
    keywordstyle        =   \bfseries,          % 关键字风格
    commentstyle        =   \rmfamily\itshape,  % 注释的风格,斜体
    stringstyle         =   \ttfamily,  % 字符串风格
    flexiblecolumns,                % 别问为什么,加上这个
    numbers             =   left,   % 行号的位置在左边
    showspaces          =   false,  % 是否显示空格,显示了有点乱,所以不现实了
    numberstyle         =   \zihao{-5}\ttfamily,    % 行号的样式,小五号,tt等宽字体
    showstringspaces    =   false,
    captionpos          =   t,      % 这段代码的名字所呈现的位置,t指的是top上面
    frame               =   lrtb,   % 显示边框
}

\lstdefinestyle{Python}{
    language        =   Python, % 语言选Python
    basicstyle      =   \zihao{-5}\ttfamily,
    numberstyle     =   \zihao{-5}\ttfamily,
    keywordstyle    =   \color{blue},
    keywordstyle    =   [2] \color{teal},
    stringstyle     =   \color{magenta},
    commentstyle    =   \color{red}\ttfamily,
    breaklines      =   true,   % 自动换行,建议不要写太长的行
    columns         =   fixed,  % 如果不加这一句,字间距就不固定,很丑,必须加
    basewidth       =   0.5em,
}
\title{
	面向对象进阶
}
\author{
	Letian Lin --- \texttt{yingziyu-Lin@outlook.com}
}

\begin{document}
\maketitle
\section{内联函数}
在成员函数前面加上inline关键字,会在编译的时候直接粘贴到目标位置,节省了时间和内存开销,但会增加编译时间。

\section{函数参数的默认值和重载}
任何有定义的表达式(如x=max(a,b))都可以作为默认值。
成员函数也可以重载,也可以带默认值参数
构造函数可以带默认值参数。在定义的时候写好默认值即可,不需要在类外边写内容的时候写初始值。
使用缺省参数的时候要注意避免函数重载的二义性。
如:
\begin{lstlisting}[language=C++]
class Location{
	...
	void valX(int val=0) {X = val};
	int valX() {return x;}
}
A.valX();//编译错误,二义性
\end{lstlisting}

\section{this指针}
指向成员函数作用的对象

非静态成员函数可以用this代表指向该函数作用对象的指针。

\section{静态成员函数和变量}
在说明前加static标识可以让其变为静态变量
静态成员变量一共就一份,所有对象共享这个变量。
sizeof不计算静态成员变量
静态成员函数也不具体作用于某个对象,也不属于某个特定的对象。用类名::成员名或常规成员访问方法访问都行。
实际上是全局变量/全局函数

静态成员函数不能访问非静态成员变量,不能调用非静态成员函数!

\section{const 标识符}
\paragraph{常量对象}在某个对象前面加const,可以定义常量对象,这种对象不能被修改。其只可以使用构造函数、析构函数、const标识的函数

\paragraph{常量方法}在函数后面加const关键字,可以将函数变为常量方法。不能修改非静态的函数值,也不能调用其他非静态函数。const类型的对象不能调用非const的成员函数。
在定义和声明常量成员函数的时候,都要加上const关键字。
\begin{lstlisting}[language=C++]
class Sample {
public:
	int value;
	void GetValue() const;
	void func() { }
};
void Sample::GetValue() const//这里不写const会出错
{
	value = 0; // wrong
	func();	   // wrong
}
int main()
{
	const Sample o;
	o.GetValue(); //
	return 0;
}
//缺少无参构造函数
\end{lstlisting}

注意,const可以构成一个重载。

同时需要注意的是,可以传const\ T\&来节约内存开销

\section{封闭类}
\subsection{定义}\paragraph{成员对象}一个类的成员变量类型是另一个类
\paragraph{封闭类}有成员对象的类叫做封闭类。
\subsection{使用}
使用封闭类的时候,需要定义构造函数。
\begin{lstlisting}[language=C++]
类名::构造函数(参数表):成员1(参数表),成员2(参数表)...
{
}
\end{lstlisting}
在使用的时候,先执行所有成员变量的生成,在执行封闭类的生成,调用次序和对象成员声明顺序一致,与在初始化列表的位置无关。
消亡时先执行封闭类的析构函数,再执行成员变量的析构函数,顺序和生成时正好相反。
const和引用的成员变量必须在成员初始化列表里面进行。

\subsection{友元}
friend关键字修饰,可以修饰类或者函数。

\end{document}
