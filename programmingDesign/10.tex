% vim:ft=tex:
%
\documentclass[UTF8]{ctexart}
\usepackage{amsmath}
\usepackage{listings}

\lstset{
    basicstyle          =   \sffamily,          % 基本代码风格
    keywordstyle        =   \bfseries,          % 关键字风格
    commentstyle        =   \rmfamily\itshape,  % 注释的风格,斜体
    stringstyle         =   \ttfamily,  % 字符串风格
    flexiblecolumns,                % 别问为什么,加上这个
    numbers             =   left,   % 行号的位置在左边
    showspaces          =   false,  % 是否显示空格,显示了有点乱,所以不现实了
    numberstyle         =   \zihao{-5}\ttfamily,    % 行号的样式,小五号,tt等宽字体
    showstringspaces    =   false,
    captionpos          =   t,      % 这段代码的名字所呈现的位置,t指的是top上面
    frame               =   lrtb,   % 显示边框
}

\lstdefinestyle{Python}{
    language        =   Python, % 语言选Python
    basicstyle      =   \zihao{-5}\ttfamily,
    numberstyle     =   \zihao{-5}\ttfamily,
    keywordstyle    =   \color{blue},
    keywordstyle    =   [2] \color{teal},
    stringstyle     =   \color{magenta},
    commentstyle    =   \color{red}\ttfamily,
    breaklines      =   true,   % 自动换行,建议不要写太长的行
    columns         =   fixed,  % 如果不加这一句,字间距就不固定,很丑,必须加
    basewidth       =   0.5em,
}
\title{
	STL
}
\author{
	Letian Lin --- \texttt{yingziyu-Lin@outlook.com}
}

\begin{document}
\maketitle
\section{string}
\subsection{初始化}有3种初始化方法。
\begin{lstlisting}
strig string1("hello"),string2(8,'x'),string3 = "hello world";
cout << string2;//xxxxxxxx
\end{lstlisting}
不允许以字符和整数来初始化字符串,但可以用字符和整数赋值给字符串,长度过长会抛出错误。
\subsection{成员函数}
s.length():返回字符串长度

s.assign(string src):将src复制到s里面。s.assign(string src,int a,int b)将src里面从a到b的字符垂直过去。

s.at(int dst):返回dst位置的字符。会进行位置检查,越界抛出异常。

s.append(string src):将src添加在s后面。s.append(string src,int a,int b):原理和assign类似。

s.compare(string src):比较s和src,大于为正,小于为负。

s.substr(int a,int b):从下标为a开始,向后b个字符的字串。

\section{iterator}
类型名::iterator it;可以看作是指针。

类型名::const_iterator it;常指针。

可以执行++操作,指向容器里面的下一个元素。如果迭代器到达了容器中的最后一个元素的后面,则迭代器变成past-the-end值,访问其是非法的。

迭代器可以由弱到强分为5类。
输入(进行只读),输出(进行读写),正向,双向,随机访问。后面的拥有前面的全部功能。

vector,deque随机迭代器,set,map,list双向迭代器,queue,stack,priority_queue不支持迭代器


\section{顺序容器}
所有都有按照字典序比较两容器的运算符,=,<,>,<=,>=,!=

empty(),max_size(),size(),swap()
begin 返回指向容器中第一个元素的迭代器
end 返回指向容器中最后一个元素后面的位置的迭代器
rbegin 返回指向容器中最后一个元素的迭代器
rend 返回指向容器中第一个元素前面的迭代器
erase 从容器中删除一个或几个元素
clear 从容器中删除所有元素

front(),back()返回一个reference

push_back(),pop_back()
\subsection{vector}
动态数组, 存取任何元素都能在常数时间完成, 在尾部增删变量
具有较佳的性能
\subsection{deque}
双端队列, 存取任何元素都能在常数时间完成, 在首尾部增删变量
具有较佳的性能

push_front(),push_back()
\subsection{list}
链表。
只能使用双向迭代器。

push_front: 在前面插入

pop_front: 删除最前面的元素

sort: 排序

remove: 删除和指定值相等的所有元素

unique: 删除所有和前一个元素相同的元素

merge: 合并两个链表, 并清空被合并的那个

reverse: 颠倒链表

splice: 在指定位置前面插入另一链表中的一个或多个元素, 并在另一链表中删除被插入的元素

\section{关联容器}
所有都有按照字典序比较两容器的运算符,=,<,>,<=,>=,!=

empty(),max_size(),size(),swap(),find(),lower_bound(),upper_bound(),count(),insert()
\subsection{set/multiset}
集合,不允许相同元素。而multiset允许相同元素。

\subsection{map/multimap}
映射,存的是key-value的值并对key排序。multimap允许相同key的存在。

\section{容器适配器}
所有都有按照字典序比较两容器的运算符,=,<,>,<=,>=,!=

empty(),max_size(),size(),swap()
\subsection{stack}
栈
\subsection{queue}
队列
\subsection{priority_queue}
优先队列

\section{算法库}
变化序列算法:
copy, remove, fill, replace, random_shuffle, swap, …
会改变容器

非变化序列算法:
adjacent-find, equal, mismatch, find, count, search,
count_if, for_each, search_n

以上函数模板都在 <algorithm> 中定义

还有其他算法, 例如 <numeric>中的算法

查找的时候要保证:a!<b&&b!<a=>a==b

需要注意的是,大部分算法库里面的算法都有两个重载,即加一个比较函数/函数对象的和不加的。

不变序列方法:不会修改算法作用的容器和对象,对于所有种类都适用,复杂度为O(n)。min_element,min,count,find_if,count,count_if,
adjacent_find,lexicographical_compare,equal,search...for_each要求f不能修改传入的容器。

变值算法:for_each,transform...

transform(it in,it out,Uop uop):对[first, last)中的每个迭代器I,
执行 uop(* I); 并将结果依次放入从 x 开始的地方
要求 uop(* I) 不得改变 * I 的值
本模板返回值是个迭代器, 即 x + (last-first)
x可以和 first相等

删除算法:将所有应当被删除的元素依次后移到序列尾,不适用于关联型容器。

变序算法:只改变顺序而不改变序列。

排序算法:一般是nlogn的。sort,stable_sort归并排序,partial_sort对区间部分排序,直到最小的n个就位,nth_element求第n大,保证前面的小于第n个小于后面的

\section{函数对象}重载了()的类的对象叫做函数对象。

常见的:(<functional>)greater/less/equal_to


\end{document}
